
\section{Definicija vlastitih funkcija}

Do sada smo već koristili mnoge funkcije koje dolaze ugrađene u Python (npr. \lstinline{print} ili \lstinline{sum}) kao i one koje su dostupne kroz module (\lstinline{math.floor}).

Funkcije su jedan od osnovnih građevnih elemenata suvremenih programskih jezika i primarna svrha im je implementacija "jedne radnje", koja može biti vrlo jednostavna, ali i vrlo kompleksna. Definicija vlastitih funkcija je stoga uobičajen postupak i služi smanjenju kompleksnosti koda time što pruža definicije potrebnih radnji koje se zatim mogu koristiti više puta. Pogledajmo prvo osnove definicije vlastitih funkcija i detalje oko postavljanja parametara, pa ćemo zatim prikazati korištenje funkcija u praktičnom primjeru.

Već smo rekli da funkcija prima nula ili više parametara, na temelju njih izvršava određen kôd te vraća rezultat. Pogledajmo kako ovo izgleda u praksi prilikom definicije jednostavne vlastite funkcije.

\begin{lstlisting}[caption={Definicija jednostavne funkcije}, label={listing:popis_redak}]
# def služi definiciji novih funkcija
def sum_two(x, y):
    # return označava kraj izvršavanja funkcije kao vrijednost koja se se smatra rezultatom
    return x + y
\end{lstlisting}

Primjer prikazuje definiciju funkcije koja prima dva parametra, zbraja ih te vraća njihov zbroj. Drugim riječima, ova funkcija odgovara operatoru \lstinline{+}. Riječ \lstinline{def} označava definiciju funkcije te se nakon nje piše naziv funkcije koji podliježe pravilima imenovanju varijabli. Nakon naziva funkcije se u oblim zagradama nabrajaju parametri funkcije. Parametri funkcije su jednostavno varijable putem kojih korisnik funkciji šalje vrijednosti potrebne za izračun. Funkcija iz primjera prima dva parametra, \lstinline{x} i \lstinline{y}, koje se unutar tijela funkcije mogu normalno koristiti kao varijable. Tijelo funkcije se podvlači pod samu liniju koja označava početak definicije kao i kod npr. kondicionala i petlji. Prije no što krenemo s primjerima korištenja funkcija u praksi, nužno je  naučiti kako se ponašaju nazivi varijabli te neke detalje oko postavljanja parametara.

\subsection{Imenski prostor}

Kôd koji sačinjava tijelo funkcije se izvršava vlastitom \textit{imenskom prostoru} odnosno nazivi varijabli se ne miješaju s nazivima varijabli izvan funkcije. Prije no što krenemo u detalje pogledajmo primjer koji prikazuje što ovo znači u praksi:

\subsection{Identifikacija parametara redoslijedom i imenom}

\subsection{Posebne vrste parametara}

\subsubsection{Niz od n parametara}

\subsubsection{Parametri s arbitrarnim imenima}


Kako bismo mogli definirati funkciju koja zbraja više od jednog broja odnosno koja oponaša već postojeću funkciju \lstinline{sum}?

def sum(numbers):
    total = 0
    for n in numbers:
        total += n
    return total

