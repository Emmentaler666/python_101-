
\section{Podaci: Vrijednosti i njihove vrste}\label{podaci_vrste}

Računalo može provoditi radnje samo s \emph{podacima}. Broj 42 je podatak, kao i 3.14. Tekst "Život, svemir i sve!" je podatak. Ove podatke ne percipiramo kao zbirke podataka pa ih uobičajeno jednostavno zovemo \emph{vrijednosti}\index{vrijednost}. Ne treba puno razmišljati kako bi shvatili da postoji beskonačno različitih vrijednosti. Također, sjetimo se sljedećeg programa:

\begin{python}{Nepoznati podatak za vrijeme izvršavanja programa}{listing:nepoznati_podatak}
text = input("Unesi tekst i pritisni <enter>: ")
print(text)
input("Pritisni <enter> za kraj")
\end{python}

U programu prikazanom u primjeru \ref{listing:nepoznati_podatak}, vrijednost varijable \mintinline{python}{text} je prilikom pisanja programa nepoznata. To je što god da je korisnik koji je pokrenuo program unio kad ga je računalo to zatražilo. Drugim riječima, ne samo da postoji beskonačno različitih vrijednosti već iste najčešće nisu poznate prilikom pisanja programa. Ovo mora biti tako: kada bi sve vrijednosti bile unaprijed zapisane u neki program, taj program bi uvijek radio istu stvar!

% todo funkcija type

U uvodu smo već uočili da postoje različite vrste vrijednosti\index{vrsta vrijednosti}. Čak i izvan sfere programiranja razlikujemo "brojke i slova". Osnovne vrste vrijednosti u kontekstu programiranja su \textbf{cijeli} i \textbf{decimalni brojevi}, \textbf{tekst}, \textbf{booleove vrijednosti} i vrijednost \textbf{None}\footnote{Koja se u mnogim drugim jezicima i bazama podataka često zove \mintinline{python}{null}.}. Ostale vrste vrijednosti možemo stvoriti koristeći se ovim osnovnim vrstama i strukturama podataka (o kojima će biti riječ kasnije). Na primjer, datum se sastoji od tri cijela broja, a možemo ga ljepše prikazati ukoliko ga pretvorimo u tekst sa znakovima ".", "-" ili "/" između znamenki. Ali čak i u tom slučaju, datum je tri broja koji ima posebna pravila prikaza kao tekst i posebna pravila za zbrajanje i oduzimanje na razini brojeva. Sve vrste vrijednosti koje nam dakle trebaju za rad s datumima su cijeli brojevi i tekst za potrebe prikaza. Osnovne vrste vrijednosti su dakle temeljne za programiranje i zapis strukturiranih podataka kao i stvaranje kompleksnijih vrsta vrijednost pa krenimo s upoznavanjem istih. 

U ovom dijelu ćemo objasniti atomske vrijednosti, odnosno cijele i decimalne brojeve, booleove vrijednosti i vrijednost \mintinline{python}{None}. To su one vrijednosti koje ne možemo raščlaniti na sastavne dijelove. Booleove vrijednosti (\mintinline{python}{True} i \mintinline{python}{False}) i vrijednost \mintinline{python}{None} su uistinu jednostavne vrijednosti i neće nam trebati puno da opišemo rad s njima. Brojevi su znatno kompleksnija tema, ali rad s njima nam je već bar donekle poznat iz matematike. S druge strane, tekst je vrlo važno podrobnije objasniti jer je znatno kompleksnija tvorevina od ostalih osnovnih vrsta vrijednosti. Također, rad s računalnim tekstom nam nije toliko poznat iz drugih područja i specifičan je za programiranje. Za razliku od brojeva, tekst slijedi druga pravila na papiru i na računalu (što uključuje i rad u softveru koji imitira papir, kao što je npr. \textit{Microsoft Word}) pa su nam pristup radu i mogućnosti manje poznate. Naravno, tekst je od vrlo velikog značaja za programiranje, World Wide Web i razmjenu podataka. Radi svega navedenog, tekstu će se posvetiti zasebno poglavlje.

\subsection{Brojevi}

% READ: https://stackoverflow.com/questions/21895756/why-are-floating-point-numbers-inaccurate

% TODO indeks cijeli broj -> vidi int
Postoje više vrsta brojeva, a one koje daleko najčešće koristimo su cijeli\index{vrsta vrijednosti!int} i decimalni brojevi\index{vrsta vrijednosti!float}. Ove vrste vrijednosti se u Pythonu nazivaju \mintinline{python}{int} i \mintinline{python}{float} što su relativno standardni nazivi za ove vrste vrijednosti u programiranju i bazama podataka. Riječ \mintinline{python}{int} je skraćeni oblik engleske riječi \textit{integer} koja doslovno znači cijeli broj. Riječ \mintinline{python}{float} dolazi od engleskog izraza \textit{floating point number} i ovaj koncept je malo problematičniji jer se ponešto razlikuje od decimalnih brojeva koji su nam poznati iz matematike s ploče u školi. Takve brojeve možemo shvatiti kao računalnu aproksimaciju decimalnih brojeva koji omogućuju varijaciju između raspona i preciznosti. Ipak, u daleko najvećem broju problema koje rješavamo programiranjem, \mintinline{python}{float} brojeve možemo shvatiti kao decimalne, a razlika nam je jedino važna ukoliko zahtijevamo preciznost na jako velikom broju decimala. Također, za razliku od jezika niže razine i baza podataka, Python nema više vrsta cijelih i decimalnih brojeva koje podržavaju različite minimalne i maksimalne veličine ili broj decimala pa u ovu temu nećemo ulaziti jer je nepotrebna za osnove programiranja i većinu programa napisanih u Pythonu\footnote{Ako pak krenete programirati, na primjer, rakete i svemirske brodove, informirajte se. Barem je jedna već pala radi pogrešnih konverzija iz jedne vrste broja u drugi, vidi na primjer \href{https://en.wikipedia.org/wiki/Ariane\_5\#Notable\_launches}{Ariane 5}.}.


%TODO sidenote 
\begin{comment} 
Programski jezici nižih razina i mnoge baze podataka imaju više vrsta cijelih i decimalnih brojeva koji se razlikuju po minimalnom i maksimalnom broju koji mogu zaprimiti, a razlog njihova korištenja odnosno razlikovanja je poboljšanje zauzeća memorije. U mnogim programskim jezicima, na primjer, riječ \mintinline{python}{int} je skraćeni naziv za \mintinline{python}{int32} odnosno 32-bitni cijeli broj koji može sadržavati vrijednosti u rasponu od -2 147 483 648 do 2 147 483 647. Svaka vrijednost vrste \mintinline{python}{int32} zauzima upravo 32 bita memorije bez obzira o kojoj se vrijednosti radi. Python, mnogi drugi programski jezici visoke razine i neke baze podataka se ne koriste ovim konceptom već automatski brinu da broj "stane" u vrstu vrijednosti \mintinline{python}{int} odnosno \mintinline{python}{float}. Ovo je dobro znati radi potencijalnog prelaska u druge tehnologije koje nisu vezane samo za programiranje već i uz baze podataka. Drugim riječima, prilikom prijelaza u druge tehnologije potrebno se informirati o vrstama vrijednosti i načinu na koji su implementirane u drugom jeziku ili softveru za baze podataka.
\end{comment}


S brojevima se najčešće radi putem aritmetičkih operatora. Slijede neki primjeri iz interaktivnog rada s Pythonom s kojima smo se već upoznali kod objašnjenja operatora, ali vrijedi ponoviti. Obzirom da su brojevi i operacije s brojevima došle "s papira", najčešće radnje s brojevima provodimo upravo kroz operatore jer nam je tako najprirodnije i najkraće pisati.


\begin{python}{Najčešće operacije s brojevima}{listing:brojevi_operacije1}
>>> 12 + 8
20
>>> 12 - 8
4
>>> 12 * 8
96
>>> 12 ** 8  # potenciranje, 12 na 8
429981696
>>> 12 / 8
1.5
>>> 12 // 8  # cjelobrojno dijeljenje
1
>>> 12 % 8   # ostatak cjelobrojnog dijeljenja
4
\end{python}


Operatore nećemo podrobnije ovdje opisivati jer su detaljnije prikazani u poglavlju \ref{radnje}. Osim standardnih matematičkih operacija s operatorima, Python ima ugrađene dodatne matematičke funkcije i konstante mnoge od kojih su dostupne kroz \textit{standardan modul} \mintinline{python}{math}\index{modul!standardan}. "Modul" je proširenje mogućnosti Pythona i možemo ga jednostavno shvatiti kao \textit{plug-in}. "Standardan" modul znači da se radi o modulu koji dolazi s Pythonom i uvijek je dostupan, odnosno koji ne zahtijeva dodatan korak instalacije modula. U primjeru \ref{listing:brojevi_operacije2} koristimo dodatne funkcije koje donosi modul \mintinline{python}{math}\index{modul!math}, a zadržimo se za sada na brojevima, dok će o modulima biti riječ u poglavlju \ref{moduli}:


\begin{python}{Još operacija s brojevima}{listing:brojevi_operacije2}
>>> round(3.1416)      # funkcija za zaokruživanje je ugrađena u Python
3
>>> round(3.1416, 2)
3.14
>>> import math        # koristiti ćemo modul math pa ga moramo uključiti
>>> math.ceil(3.1416)  # zaokruži na prvi viši cijeli broj, obratno je "floor"
4
>>> math.sqrt(625)     # drugi korijen
25.0
>>> math.sin(45)       # sinus; uključene su i ostale trigonometrijske funkcije
0.8509035245341184
>>> math.pi
3.141592653589793      # konstanta $\pi$
\end{python}


Modul \mintinline{python}{math} donosi i mnoge druge mogućnosti, a postoje i kompleksna proširenja Pythona orijentirana na rad s brojevima bilo da se radi o naprednijoj matematici ili specijalizacijama poput statistike. Obzirom da se ovdje u načelu radi ili o poznatim nam matematičkim postupcima i funkcijama (u slučaju modula \mintinline{python}{math}, to su mogućnosti poput korjenovanja i trigonometrijskih funkcija) ili pak o naprednim područjima nevezanim za osnove programiranja (npr. statistika, strojno učenje) no o novim konceptima ovdje možemo stati s prikazom rada s brojevima i prikazati druge vrste vrijednosti koje su specifičnije za programiranje.


\subsection{Booleove vrijednosti}

% osnova bool vrijednosti
Različitih brojeva i tekstualnih nizova ima beskonačno, ali samo su dvije booleove vrijednosti i označavaju "da" i "ne", odnosno istinu i neistinu, postoji i ne postoji, ima struje i nema struje, 1 i 0 ... Ove dvije vrijednosti su neobično važne za računarstvo jer su to binarne znamenke koje u kontekstu suvremenih digitalnih elektroničkih računala tvore binarni brojevni sustav putem kojeg kodiramo bilo koje podatke za potrebe interakcije s računalnim hardverom odnosno za potrebe pohrane u memoriju i izvršavanje instrukcija. U programiranju, ove dvije vrijednosti imaju i širu logičku upotrebu, kao što ćemo uskoro vidjeti u primjerima. U Pythonu ih nazivamo \mintinline{python}{True} i \mintinline{python}{False}.

\subsubsection{Usporedba brojeva}

% bool vrijednosti i operatori
Do bool vrijednosti u programima često dolazimo posebnim operatorima i metodama za usporedbu. Ovakvi operatori i metode služe provjeri istinitosti neke tvrdnje odnosno odgovaraju na da-ne pitanja. Pogledajmo prvo najčešće operatore za usporedbu brojeva:

\begin{python}{Usporedba brojeva}{listing:bool1}
>>> x = 2   # ovo je pridruživanje varijabli, a ne provjera jednakosti
>>> y = 3
>>> x < y   # x je manje od y
True
>>> x >= y  # x je veće ili jednako y
False
>>> x == y  # x je jednako y
False
>>> x != y  # x je različit od y
True
>>> x + y == 5  # rezultat izraza "x + y" iznosi 5
True
>>> x + (y == 5)  # što se ovdje dogodilo ???
2
\end{python}

Neke detalje smo spomenuli već u uvodu i podrobnije  objasnili u poglavlju \ref{radnje}, ali vrijedi ponoviti. Primijetimo operator \mintinline{python}{==} u izrazu \mintinline{python}{x == y}. Rekli smo da operator \mintinline{python}{=}\index{operator!pridruživanje} znači pridruživanje vrijednosti varijabli. Bilo bi višeznačno koristiti isti znak za operator koji provjerava jednakost pa za to postoji poseban operator: \mintinline{python}{==}\index{operator!usporedba}. 

Na primjer, u matematici:

\begin{python}{Znak "=" u matematici}{listing:bool2}
x = 1      # (pridruživanje: neka x bude 1)
x + 1 = 2  # (tvrdnja jednakosti: x + 1 je jednako 2)
\end{python}

U Pythonu:

\begin{python}{Operatori "=" i "==" u programiranju}{listing:bool3}
>>> x = 1       # pridruživanje: "neka x bude 1", ovo nije izraz pa nema rezultata
>>> x + 1 == 2  # provjera jednakosti: "da li se izraz "x + 1" evaluira u 2"
True
\end{python}

A kako to da je izraz \mintinline{python}{x + y == 5} različit od izraza \mintinline{python}{x + (y == 5)}? Prvi dio odgovora leži u redoslijedu izvršavanja operacija. Kao što je opisano u poglavlju \ref{radnje}, prvo se provode aritmetičke operacije pa zatim operacije usporedbe. U izrazu \mintinline{python}{x + y == 5} prvo se provodi zbrajanje pa tek zatim usporedba. Taj izraz je, dakle, isto što i \mintinline{python}{(x + y) == 5}. U izrazu \mintinline{python}{x + (y == 5)} smo zagradama promijenili redoslijed izvršavanja operacija: prvo se provodi usporedba \mintinline{python}{y == 5} koja vraća rezultat \mintinline{python}{False}. Zatim se provodi zbrajanje, a u ovom slučaju to je izraz \mintinline{python}{x + False}. False je ekvivalentan vrijednosti 0 pa je taj izraz isto što i \mintinline{python}{x + 0} te iznosi vrijednosti varijable \mintinline{python}{x}, odnosno \mintinline{python}{2}. Za redoslijed izvršavanja operatora znamo kako iz prijašnjeg poglavlja tako i iz osnova matematike. Za vrijednosti True i False vrijedi zapamtiti da su ekvivalentne vrijednostima \mintinline{python}{1} i \mintinline{python}{0} i da mnogi jezici dopuštaju aritmetičke operacije s njima. Navedeno je pobliže prikazano u primjeru \ref{listing:bool_je_broj}.

\begin{python}{Booleove vrijednosti su (skoro) isto što i 0 i 1}{listing:bool_je_broj}
>>> n = 42
>>> n + True  # True je jednak 1
43
>>> n - True
41
>>> n + False  # False je jednak 0
42
>>> n * False
0
>>> True == 1
True
>>> True is 1  # ali True je u memoriji različita vrijednost od 1
False
>>> True is not 1
True    
\end{python}


\subsubsection{Usporedba drugih vrsta vrijednosti}

Situacije koje provjeravamo s očekivanim odgovorima "da" i "ne" ovise najviše o vrstama vrijednosti. Za brojeve se odnose na matematičke koncepte (veće, manje, jednako \textellipsis), ali što je na primjer s tekstom? Tekst podržava mnoštvo tvrdnji koje možemo provjeravati, pogledajmo neke od njih:

\begin{python}{Usporedba teksta}{listing:bool4}
>>> a = 'nešto'
>>> b = 'nešto drugo'
>>> a == b     #  većina osnovnih vrsti podataka dopušta provjeru jednakosti
False
>>> 'd' in a   # da li se znak 'd' nalazi u tekstu a
False
>>> 'dr' in b  # da li se znakovi 'drugo' nalaze u tekstu b
True
>>> b.startswith(a)  # posebne provjere se često provode posebnim metodama za vrste vrijednosti
True
>>> s = '122'
>>> s.isdigit()      # da li se tekst sastoji samo od brojeva?
True
>>> s = 'abc'        # da li se tekst sastoji samo od slova?
>>> s.isalpha()
True
>>> s = 'abc4'
>>> s.isalpha()
False
\end{python}

\subsubsection{Koje vrijednosti postoje?}

% bool(something)
Također, bilo koju vrijednost možemo reducirati na bool vrijednost odnosno na "postoji/ne-postoji". Velika većina vrijednosti se procjenjuje kao \mintinline{python}{True} jer se odnosi na vrijednost koju smatramo "postojećom". Kod brojeva je ovdje očito o čemu se priča, 0 se procjenjuje kao \mintinline{python}{False}, a svi ostali brojevi kao \mintinline{python}{True}. Dapače, već smo vidjeli su vrijednosti \mintinline{python}{True} i \mintinline{python}{False} ekvivalentne brojevima 1 i 0 te da je čak s njima i moguće provoditi aritmetičke operacije. Kod ostalih vrijednosti, međutim, također postoje "prazne vrijednosti" koje se smatraju \mintinline{python}{False}. Riječ je u načelu o praznim "zbirkama": tekst od 0 znakova, popis s 0 elemenata itd. S mnogim zbirkama, odnosno \emph{strukturama podataka}, ćemo se sresti u idućim poglavljima jer su strukture podataka nezaobilazna tema za programiranje.

\begin{python}{Koje vrijednosti "postoje"?}{listing:bool5}
>>> bool(42)
True
>>> bool(0)
False
>>> bool(None) # vrijednost None je opisana u nastavku ovog poglavlja
False
>>> bool('neki tekst')  # tekst koji sadrži barem jedan znak
True
>>> bool('')   # prazan tekstovni niz, tekst od nula znakova
False
>>> bool([])   # prazan popis, više o popisima kasnije
False
# ali zapamtimo da se prazne zbirke procjenjuju kao False
\end{python}


\subsubsection{Booleovi operatori}

Također, booleove vrijednosti se mogu kombinirati booleovim operatorima \mintinline{python}{and} i \mintinline{python}{or}. Bool vrijednost se može i negirati, odnosno pretvoriti u suprotnu bool vrijednost pomoću operatora \mintinline{python}{not} koji prethodi varijabli, odnosno vrijednosti. Na primjer:

\begin{python}{Booleovi operatori}{listing:bool6}
>>> True and False
False
>>> True or False
True
>>> not False
True
>>> True and not False
True

# shodno gore navedenom:
>>> x = 1
>>> y = 2
>>> x + y == 3 and x == 1
True
>>> x + y == 1000 and x == 1
False
>>> x + y == 1000 or y == 2
True
>>> not x == 1
False

>>> a = 'nešto'
>>> b = 'nešto drugo'
>>> a.startswith('n') and b.startswith('n')
True
>>> a.startswith('n') and b.startswith('x')
False
>>> a.startswith('n') or b.startswith('x')
True
>>> a.startswith('n') and not b.startswith('x')
True
\end{python}


\subsection{None}

\mintinline{python}{None} je jedinstvena vrijednost koja označava nepostojanje vrijednosti. Drugim riječima, postoji samo jedna moguća vrijednost čija vrsta je \mintinline{python}{None} i ta vrijednost je \mintinline{python}{None}. Možda zvuči čudno, ali ova vrijednost je vrlo korisna kada je potrebno eksplicitno zapisati da neka varijabla "nema vrijednost".

Promotrimo, na primjer, razliku između vrijednosti \mintinline{python}{None} i vrijednosti \mintinline{python}{0}.

\begin{python}{Čemu služi vrijednost None?}{listing:none}
broj_knjiga = 0
broj_knjiga = None
\end{python}

U slučaju \mintinline{python}{broj_knjiga = 0} znači poznatu informaciju: "nula knjiga". Kada bi se ovaj podatak koristio, na primjer, za broj posuđenih knjiga, vrijednost 0 bi značila da nema posuđenih knjiga. S druge strane, vrijednost \mintinline{python}{None} kod \mintinline{python}{broj_knjiga = None}, pak, znači: "broj knjiga je nepoznat". Vrijednost \mintinline{python}{None} posebno je korisna za rad s podacima i dizajn vlastitih funkcija (što je prikazano kasnije). U drugim jezicima i bazama podataka, vrijednost \mintinline{python}{None} se često naziva \mintinline{python}{null}. 

Za razliku od brojeva, teksta i booleovih vrijednosti, s vrijednosti \mintinline{python}{None} nemamo posebne radnje jer za ovu vrstu vrijednosti nemaju smisla.
