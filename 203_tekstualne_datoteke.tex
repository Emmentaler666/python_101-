
\section{Rad s tekstualnim datotekama}

Rad s običnim tekstualnim datotekama važniji je no što se to možda čini na prvi pogled. Na primjer, svaki program nastaje kao običan tekst, cijeli WWW se zasniva na običnom tekstu, kao i razmjena podataka među informacijskim sustavima. Čitanje i pisanje tekstualnih datoteka je stoga neobično važno ne samo za programiranje već i za sveukupno računarstvo.

Glavni mehanizam za čitanje i pisanje tekstualnih datoteka u Pythonu je funkcija \mintinline{python}{open}. Ovisno o parametrima, ova funkcija pristupa nekoj datoteci za potrebe čitanja iz nje ili pisanja u nju. Za razliku od istoimene mogućnosti u programima s grafičkim sučeljem, dakle, funkcija \mintinline{python}{open} samo rezervira određenu datoteku za pristup, a ne usnimava sadržaj datoteke. Radi toga, ova funkcija služi i pisanju u datoteke, a čak i stvaranju novih tekstualnih datoteka. Bez obzira da li čitali ili pisali, nakon što se izvrše potrebne radnje s otvorenom datotekom, istu je potrebno zatvoriti putem metode \mintinline{python}{close}. Pogledajmo kako zapisati neki tekst u novu datoteku:

\pythonfile{Zapisivanje teksta u novu datoteku}{listing:tekst_zapis}{code_python_osnove/text_files/files_a.py}

Funkcija \mintinline{python}{open} prima putanju do datoteke kao prvi parametar, mod rada kao drugi parametar i dodatne parametre od kojih je vrlo korisna specifikacija kodne stranice putem opcionalnog parametra \mintinline{python}{encoding}. Dapače, preporučeno je uvijek definirati \mintinline{python}{encoding} prilikom čitanja i pisanja jer u tom slučaju jasno kontroliramo kodnu stranicu našeg teksta i time sprječavamo gubitak slova i ostalih znakova. U tom smislu, kodna stranica "utf-8" je danas \textit{de-facto} standard.

Putanja može biti apsolutna e.g. ("c:/direktorij/datoteka.txt") ili relativna ("datoteka.txt" ili "direktorij/datoteka.txt"). Ukoliko je putanja relativna, kao u našem primjeru, smatra se da je relativna od direktorija u kojem se nalazi .py datoteka koju smo pokrenuli. Na primjer, putanja "datoteka.txt" se odnosi na datoteku u istom direktoriju u kojem je i pokrenuta .py datoteka. Putanja "neki\_direktorij/datoteka.txt" se odnosi na datoteku "datoteka.txt" koja se nalazi u direktoriju "neki\_direktorij" koji se pak nalazi u istom direktoriju u kojem je i pokrenuta .py datoteka. U ranijem primjeru, u istom direktoriju će nam se pojaviti datoteka "papiga.txt" s tekstom koji smo naredili zapisati.

Funkicija \mintinline{python}{open}, može čitati postojeće datoteke, dodavati u postojeće ili stvarati nove. To kontroliramo putem parametra \mintinline{python}{mode} koji možemo shvatiti kao "mod rada" funkcije \mintinline{python}{open}. Pogledajmo kako dodati neke retke u "papiga.txt" datototeku.

\pythonfile{Dodavanje teksta u postojeću datoteku}{listing:tekst_dodavanje}{code_python_osnove/text_files/files_b.py}

Korisni modovi za \mintinline{python}{open} su:

\begin{itemize}
\item \textbf{r} - čitaj; \textit{default}
\item \textbf{w} - stvori i piši; ako datoteka postoji, obriši sadržaj
\item \textbf{x} - stvori i piši; ako datoteka postoji, javi grešku
\item \textbf{a} - stvori i piši; ako datoteka postoji, nastavi pisati na kraj
\end{itemize}

Do sada smo samo pisali u datoteku. Čitanje je jednostavno drugi način operacije funkcije \mintinline{python}{open}. Čitanje je zapravo zadani (eng. \textit{default}) način rada ove funkcije, ali u ovom poglavlju smo prvo stvorili novu datoteku kako bi imali što čitati. Također, parametar \mintinline{python}{"r"} zapravo ne treba navoditi jer se podrazumijeva, ali njegovim navođenjem se mod rada jasnije vidi.


\pythonfile{Čitanje teksta iz postojeće datoteke}{listing:tekst_citanje}{code_python_osnove/text_files/files_c.py}


\begin{pythonp}{\ref{listing:tekst_citanje}}    
This parrot is no more!
He has ceased to be!
'E's expired and gone to meet 'is maker!
'Is metabolic processes are now 'istory!
'E's off the twig!
..... THIS IS AN EX-PARROT!!
\end{pythonp}


Pogledajmo kako se petlja \mintinline{python}{for} može koristiti u kontekstu čitanja teksta iz datoteke:


\pythonfile{Prebiranje po linijama postojeće datoteke}{listing:tekst_prebiranje}{code_python_osnove/text_files/files_d.py}


\begin{pythonp}{\ref{listing:tekst_prebiranje}}
1 This parrot is no more!
2 He has ceased to be!
3 'E's expired and gone to meet 'is maker!
4 'Is metabolic processes are now 'istory!
5 'E's off the twig!
6 ..... THIS IS AN EX-PARROT!!
\end{pythonp}

Drugim riječima, po otvorenoj datoteci se može iterirati po recima teksta. Na ovaj način možemo raditi s datotekama bilo koje veličine pa čak i onima koje nam ne stanu u memoriju jer u svakom trenutku imamo samo jedan redak u memoriji, a ne cijeli tekstualni sadržaj neke datoteke.


\subsection{Naredba \mintinline{python}{with}}

Što će se dogoditi s datotekom ako slučajno ne pozovemo naredbu \mintinline{python}{close}? Postoji šansa da se u datoteku nije zapisao sav tekst i da ona u operacijskom sustavu ostane "rezervirana za pristup". Problem s do sada prikazanim načinom zatvaranja datoteke nije samo u tome da možemo zaboraviti provesti naredbu \mintinline{python}{close}, već se može dogoditi da se ona ne provede radi ranije greške. Pogledajmo primjer:


\pythonfile{Greška prije pozivanja naredbe \mintinline{python}{close}}{listing:tekst_greska}{code_python_osnove/text_files/files_e.py}


Kako riješiti da se naredba \mintinline{python}{close} uvijek izvrši, bez obzira na potencijalne ranije greške u kôdu? Iz onoga što do sad znamo, mogli bismo probati s naredbom \mintinline{python}{try} i ukoliko iskoristimo i komponentu \mintinline{python}{finally} u tome bi i uspjeli\footnote{Za vježbu razmislite kako bismo naredbom \mintinline{python}{try} mogli \textit{garantirati} izvršavanje pozivanje metode \mintinline{python}{close} čak i u slučaju ranije pogreške}. Ipak, ovakvo rješenje se smatra nezgrapnim i nije namijenjeno korištenje naredbe \mintinline{python}{try}. Radi ovakvih i sličnih slučajeva se u noviji Python dodala naredba \mintinline{python}{with} koja je postala idiom upravo za otvaranje i zatvaranje datoteka kao i slične situacije.


\pythonfile{Rad s tekstualnim datotekama pomoću naredbe \mintinline{python}{with}}{listing:tekst_with}{code_python_osnove/text_files/files_with.py}


Radi elegantnosti izvedbe i vezanu sigurnost, preporuča se koristiti naredbu \mintinline{python}{with}. Prikazani način pristupa tekstualnim datotekama je stoga idiom u novijem Pythonu, ali ne možemo u potpunosti shvatiti kako funkcionira bez razumijevanja koncepata "otvaranja" i "zatvaranja" datoteka. Također, postoje slučajevi u kojima je korištenje naredbe \mintinline{python}{with} nezgrapno. U tim slučajevima možemo nastaviti koristiti metodu \mintinline{python}{close}.
